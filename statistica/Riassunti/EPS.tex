\documentclass[12pt,a4paper, openany]{book}

% Pacchetti utili
\usepackage{amsmath,amssymb}
\usepackage[utf8]{inputenc} % codifica UTF-8
\usepackage[T1]{fontenc}
\usepackage{lmodern}
\usepackage{graphicx}       % per inserire loghi/immagini
\usepackage{setspace}       % per spaziatura
\usepackage{geometry}       % margini
\usepackage[export]{adjustbox}
\geometry{top=3cm,bottom=3cm,left=3cm,right=3cm}
\begin{document}
	\begin{titlepage}
		\centering
		% Logo in alto (opzionale)
		% \includegraphics[width=0.25\textwidth]{logo.png}\par\vspace{1cm}
		
		{\Large \textsc{Università degli Studi di Torino}}\\[0.5cm]
		{\large Dipartimento di Informatica}\\[3cm]
		
		% Titolo
		{\Huge \bfseries Appunti di probabilità
			e statistica}\\[0.5cm]
		%{\LARGE Struttura, Progettazione e Implementazione}\\[2cm]
		
		% Autore
		{\large Autore}\\
		{\Large Francesco Martino}\\
		{\Large Riccardo Marengo}\\[1.5cm]
		
		% Data
		{\large Anno Accademico 2025--2026}
		
		\vfill
		
	\end{titlepage}
	\chapter{Teoria}
	\section{Elementi di Probabilità e Statistica: Lezione 1 del 16-09-2025}
	
	\subsection{Introduzione alla probabilità}
	
	La probabilità riguarda qualsiasi fenomeno non facilmente prevedibile
	(lancio di dadi o monete, tempi di attesa, eccetera).
	Il termine che descrive la proprietà di queste situazioni è aleatorietà.
	Il fatto che questi fenomeni non siano facilmente prevedibili non vuol dire
	che siano effettivamente casuali, ma solo che non abbiamo tutte le
	informazioni necessarie per poterle prevedere accuratamente.
	
	DEF: Un insieme è una collezione di oggetti denominati elementi
	dell'insieme. Un insieme è denominato da una lettera latina
	maiuscola e i suoi elementi sono elencati esplicitamente tra parentesi graffe
	o determinati con una proprietà.
	
	Esempio:
	
	\[
	S = \{x_1, x_2, x_3\} = \{x_2, x_1, x_3\} = \{x_1, x_1, x_1, x_2, x_3\}
	\]
	
	\[
	S = \{x \mid P(x) è soddisfatta
	\]
	\begin{math}
		[
		\{x \in
		\mathbb{N}
		\mid x \leq 3\} = \{0, 1, 2, 3\}
		4]
	\end{math}
	(anche se i numeri naturali sono a volte definiti a partire da 1)
	
	Altre definizioni:
	\begin{itemize}
		\item La cardinalità dell'insieme $S$ è il numero $|S|$, finito o infinito, di elementi di $S$.
		\item Un insieme $S$ è numerabile $\Leftrightarrow$ esiste una funzione iniettiva $f: S \to \mathbb{N}$.
	\end{itemize}
	
	Esistono le operazioni di unione, intersezione, ecc. e le relazioni di inclusione, inclusione stretta, ecc.
	
	L'insieme che non contiene alcun elemento è detto \emph{insieme vuoto} e si scrive $\varnothing$.
	
	DEF: Lo spazio campionario, di solito indicato come $\Omega$, è l'insieme che contiene tutti i possibili esiti di un esperimento probabilistico.
	
	Esempio: Se l'esperimento consiste nel lancio di un dado, lo spazio campionario è:
	
	\[
	\Omega = \{\text{faccia 1}, \text{faccia 2}, \ldots, \text{faccia 6}\}
	\]
	
	\section{Elementi di Probabilità e Statistica: Lezione 2 del 18-09-2025}
	
	\subsection{Modello probabilistico}
	
	Un modello probabilistico è un oggetto matematico composto da:
	\begin{itemize}
		\item uno spazio campionario $\Omega$, cioè l'insieme che contiene tutti i possibili esiti di un esperimento,
		\item una funzione che assegna a ogni elemento dell'insieme delle parti di $\Omega$ un numero reale.
	\end{itemize}
	
	Esempio: determinare la probabilità che il lancio di un dado risulti in un numero pari.
	\[
	P(\{2,4,6\}) = \; ?
	\]
	
	La funzione $P$ deve avere alcune proprietà fondamentali:
	\begin{itemize}
		\item per ogni sottoinsieme $S \subseteq \Omega$, vale $0 \leq P(S) \leq 1$;
		\item $P(\Omega) = 1$;
		\item se $A$ e $B$ sono sottoinsiemi disgiunti di $\Omega$, allora
		\[
		P(A \cup B) = P(A) + P(B).
		\]
	\end{itemize}
	\section*{Lezioni di EPS del 19/09/2025 }
	
	
	\section*{Lezioni di EPS del 23-24/09/2025 }
	
	\subsection*{Proprietà di $\mathbb{P}$ derivate dagli assiomi }
	Siano A, B, C eventi .
	\begin{enumerate}
		\item $A \subseteq B \Rightarrow \mathbb{P}(A) \leq \mathbb{P}(B)$ 
		\item $\mathbb{P}(A \cup B) = \mathbb{P}(A) + \mathbb{P}(B) - \mathbb{P}(A \cap B)$ 
		\item $\mathbb{P}(A \cup B) \leq \mathbb{P}(A) + \mathbb{P}(B)$ (sub-additività) 
	\end{enumerate}       
	
	\section*{Probabilità condizionata }
	Vorremmo calcolare delle probabilità che tengono conto di "informazioni parziali" sull'esito dell'esperimento .
	
	\textbf{ESEMPIO} Lancio di 2 dadi equi a 6 facce . Questo esperimento equivale a lanciare due dadi in sequenza solo se assegnamo un ordinamento ai due dadi lanciati (quindi per esempio consideriamo $(2, 5)$ diverso da $(5, 2)$) .
	$$ \Omega = \{ w | w \in \{1, \dots, 6\}^2\} \Rightarrow |\Omega| = 6^2 = 36 $$ 
	Gli esiti dell'esperimento sono vettori di dimensione 2, per cui possono essere rappresentati graficamente come punti su un piano . La funzione probabilità $\mathbb{P}$ è uniforme discreta . Vediamo le probabilità di due eventi e la loro intersezione :
	
	\begin{itemize}
		\item $A = "\text{la somma dei dadi è 9}" = \{w \in \Omega | w_1 + w_2 = 9\} = \{(4, 5), (5, 4), (3, 6), (6, 3)\}$  $\Rightarrow \mathbb{P}(A) = 4/36 = 1/9$ 
		\item $B = "\text{il primo dado vale 6}" = \{w \in \Omega | w = (6, j) \text{ con } j \in (1, \dots, 6)\}$  $\Rightarrow |B| = 6 \Rightarrow \mathbb{P}(B) = 6/36 = 1/6$ 
		\item $A \cap B = \{(6, 3)\}$  $\Rightarrow |A \cap B| = 1 \Rightarrow \mathbb{P}(A \cap B) = 1/36$ 
	\end{itemize}
	Come possiamo calcolare la probabilità che si verifichi $A$ presupponendo che $B$ si sia già verificato? Dobbiamo introdurre un nuovo concetto :
	
	\textbf{DEFINIZIONE} (probabilità condizionata) 
	Dato $(\Omega, \mathcal{P}(\Omega), \mathbb{P})$ spazio di probabilità e $B$ evento t.c $\mathbb{P}(B) \neq 0$ ;
	Definiamo probabilità condizionata rispetto a $B$, indicata con $\mathbb{P}(\cdot | B) : \mathcal{P}(\Omega) \rightarrow \mathbb{R}$ t.c.
	$$ \mathbb{P}(A|B) = \frac{\mathbb{P}(A \cap B)}{\mathbb{P}(B)} $$ 
	
	Nell'esempio di prima,
	$$ \mathbb{P}(A|B) = \frac{\mathbb{P}(A \cap B)}{\mathbb{P}(B)} = \frac{(1/36)}{(1/6)} = 6/36 = 1/6 $$ 
	Se $C = "\text{il primo dado ha dato 1}"$, quanto vale $\mathbb{P}(A|C)$?  Sappiamo che $\mathbb{P}(A|C) = \mathbb{P}(A \cap C) / \mathbb{P}(C)$ . Ma $A \cap C = \emptyset$ perché non esistono coppie della forma $(1, j)$ nell'insieme $A$ . Allora $\mathbb{P}(A \cap C) = 0 \Rightarrow \mathbb{P}(A|C) = 0 / \mathbb{P}(C) = 0$ .
	
	Notiamo che le funzioni $\mathbb{P}(\cdot | B)$ e $\mathbb{P}(\cdot | C)$ non sono più uniformi . Dimostriamo però che le funzioni di probabilità condizionata conservano le proprietà delle funzioni di probabilità .
	
	\textbf{DIMOSTRAZIONE} 
	Dato $(\Omega, \mathcal{P}(\Omega), \mathbb{P})$, dobbiamo dimostrare che $\forall B \in \mathcal{P}(\Omega)$, $\mathbb{P}(\cdot | B)$ soddisfa: 
	\begin{enumerate}
		\item $\forall A \in \mathcal{P}(\Omega), \mathbb{P}(A|B) \geq 0$ 
		\item $\mathbb{P}(\Omega|B) = 1$ 
		\item $A_1, A_2$ eventi disgiunti $\Rightarrow \mathbb{P}(A_1 \cup A_2|B) = \mathbb{P}(A_1|B) + \mathbb{P}(A_2|B)$ 
	\end{enumerate}
	
	\textit{Prova:}
	\begin{enumerate}
		\item Per definizione di $\mathbb{P}$ sappiamo che $\mathbb{P}(B), \mathbb{P}(A \cap B) \geq 0$ 
		allora anche $\mathbb{P}(A|B) = \frac{\mathbb{P}(A \cap B)}{\mathbb{P}(B)} \geq 0$ .
		\item Per definizione di $\mathbb{P}$ sappiamo che $\mathbb{P}(\Omega) = 1 \Rightarrow \mathbb{P}(\Omega|B) = \frac{\mathbb{P}(\Omega \cap B)}{\mathbb{P}(B)} = \frac{\mathbb{P}(B)}{\mathbb{P}(B)} = 1$\newline
		$\mathbb{P}$ ($\Omega$|B) = $\frac{\mathbb{P}(\Omega \cap|B}{}$
		\item $\mathbb{P}(A_1 \cup A_2|B) = \frac{\mathbb{P}((A_1 \cup A_2) \cap B)}{\mathbb{P}(B)}$ 
		$$ = \frac{\mathbb{P}((A_1 \cap B) \cup (A_2 \cap B))}{\mathbb{P}(B)} $$ 
		Poiché $A_1$ e $A_2$ sono disgiunti, anche $(A_1 \cap B)$ e $(A_2 \cap B)$ lo sono.
		$$ = \frac{\mathbb{P}(A_1 \cap B) + \mathbb{P}(A_2 \cap B)}{\mathbb{P}(B)} $$ 
		$$ = \mathbb{P}(A_1|B) + \mathbb{P}(A_2|B) $$ 
	\end{enumerate}
	(1), (2) e (3) sono verificate $\Rightarrow \mathbb{P}(\cdot | B)$ è una funzione probabilistica $\square$ .
	
	
	
	
	
	
	
	
	
	
	
	
	
	
	
	
	
	
	
	
	
	%%%%%%%%%%%%%%%%%%%%%%%%%%%%%%%%%%%%%%%%%%%%%%%%%%%%%%%%%%%%%%%%%%%%%%%%%%55
	%%%%%%%%%%%%%%%%%%%%%%%%%%%5ESERCIZI
	\chapter{Esercizi ed esempi}
	\subsection*{Esercizio 1 - biglie }
	Una scatola contiene 3 biglie: una rossa (R), una verde (V), una blu (B) . Considero il seguente esperimento: Estraggo una biglia, la reimbussolo e ne estraggo una seconda . Scrivere un possibile modello probabilistico e calcolare la probabilità che le due biglie estratte siano uguali .
	\begin{itemize}
		\item $\Omega = \{w = (w_1, w_2) \text{ con } w_i \in \{R, V, B\}, i \in \{1, 2\}\}$ 
		\item $|\Omega| = 3^2 = 9$ 
		\item $\mathbb{P}(A) = \frac{|A|}{|\Omega|}$ (probabilità uniforme discreta) 
	\end{itemize}
	Se $A = \{(R, R), (V, V), (B, B)\}$ ("le biglie estratte sono uguali")  $\Rightarrow \mathbb{P}(A) = \frac{|A|}{|\Omega|} = 3/9 = 1/3$ .
	
	\subsection*{Esercizio 2 - dadi }
	Un dado a 4 facce equo viene lanciato ripetutamente fino all'uscita di un numero pari . Scrivere lo spazio campionario per questo esperimento . Quanti sono gli esiti possibili? Posso usare la legge uniforme discreta? 
	
	Gli esiti possibili sono tutti gli esiti che contengono zero o più numeri dispari seguiti da esattamente un numero pari $((2), (1, 1, 2), (1, 3, 1, 4), \text{ ecc.})$ .
	$$ \Omega = \{w = (w_1, \dots, w_n) \text{ con } n \in \{1, 2, \dots\} | w_i \text{ è pari } \Leftrightarrow i = n\} $$ 
	La cardinalità di $\Omega$ è $\infty$  quindi non possiamo usare la legge uniforme discreta .
	
	\subsection*{Esercizio 3 - amici }
	Un gruppo di amici (5 ragazzi e 10 ragazze) viene messo in fila in ordine causale .
	\begin{itemize}
		\item $\Omega = \{w = (w_1, \dots, w_{15}) | w_i \text{ è una persona nel gruppo e } i \neq j \Rightarrow w_i \neq w_j\}$ 
		\item $|\Omega| = 15!$ 
		\item Possiamo usare la legge uniforme discreta .
	\end{itemize}
	\begin{enumerate}
		\item Calcolare la probabilità che la persona in quarta posizione sia un ragazzo . \\
		$A = "\text{w4 è un ragazzo}" \Rightarrow |A| = 5 \cdot 14!$ 
		$$ \mathbb{P}(A) = \frac{5 \cdot 14!}{15!} = \frac{5 \cdot 14!}{15 \cdot 14!} = 5/15 = 1/3 $$ 
		\item Calcolare la probabilità che la persona in dodicesima posizione sia un ragazzo . \\
		$B = "\text{w12 è un ragazzo}" \Rightarrow |B| = |A|$ 
		$$ \mathbb{P}(B) = \mathbb{P}(A) = 1/3 $$ 
		\item Calcolare la probabilità che Luigi sia in terza posizione . \\
		$C = "\text{w3 = Luigi}" \Rightarrow |C| = 14!$ 
		$$ \mathbb{P}(C) = \frac{14!}{15!} = 1/15 $$ 
	\end{enumerate}
	
	\subsection*{Esercizio 4 - giuria }
	Una giuria composta da 5 persone viene selezionata da 6 uomini e 9 donne . Con quale probabilità la giuria sarà composta da 3 uomini e due donne .
	\begin{itemize}
		\item $\Omega = \{\text{sottoinsiemi } S \text{ delle 15 persone } | |S| = 5\}$ 
		\item $|\Omega| = \frac{15!}{5! \cdot 10!} = \frac{15 \cdot 14 \cdot 13 \cdot 12 \cdot 11}{5!} = 3003$ 
	\end{itemize}
	\textit{Per sapere se $\mathbb{P}$ è uniforme discreta, bisogna pensare se gli elementi di $\Omega$ sono finiti e hanno tutti la stessa probabilità di verificarsi.} 
	
	$A = "\text{estraiamo due donne e tre uomini}"$ 
	$$ |A| = \binom{6}{3} \cdot \binom{9}{2} = \frac{6!}{3!3!} \cdot \frac{9!}{2!7!} $$
	$$ |A| = \frac{6 \cdot 5 \cdot 4 \cdot 9 \cdot 8}{3 \cdot 2 \cdot 2} $$ 
	$$ \mathbb{P}(A) = 720 / 3003 \approx 0.2398 $$ 
\end{document}